\section{Usage}

\subsection{Structure of \TeX\ Files}

The basic usage is as follows: consider a main document that includes files for
individual chapters, named for example \texttt{main.tex}:


\begin{verbatim}
\documentclass{article}
\author{John Doe}
\title{Sample Document}
\usepackage{upquote}
\usepackage{microtype}
\usepackage{hyperref}
\begin{document}
\maketitle
\tableofcontents
% !TEX root = texblend-doc.tex
% !TEX TS-program = lualatex

hello world

\section{Usage}



\subsection{HTML Support}

\TeX4ht note about the \verb|--texoptions| command line option: you can use
this option to pass all command line arguments to \verb|make4ht|. You may need
to include the \verb|-| character in the place of filename for the correct
execution.

For example, if you want to use the \verb|fn-in| option, use the following command:

\begin{verbatim}
$ texblend  -H -O '- "fn-in"' sample.tex
\end{verbatim}

If you want to use for example a configuration file, use

\begin{verbatim}
$ texblend -l -H -O '-c config.cfg - "fn-in"' sample.tex
\end{verbatim}

\end{document}
\end{verbatim}

We can compile this document to obtain a PDF file containing the text of all
chapters. However, by using \TeX Blend, we can generate separate PDFs for each
chapter. The advantage is that all commands defined in the packages used in the
main document will be available.

In the included files, we can use special directives that indicate the main
file and the engine to be used for compilation. Similar directives are employed
by various \TeX editors, such as \TeX Shop or \TeX Works.

For instance, a file like \texttt{intro.tex}, which is included in the main document,
might look like this:

\begin{verbatim}
% !TEX root = main.tex
% !TEX TS-program = lualatex

\section{Introduction}
This tool compiles individual files that are included 
as parts of larger documents. 
\end{verbatim}



\subsection{HTML Support}

\TeX4ht note about the \verb|--texoptions| command line option: you can use
this option to pass all command line arguments to \verb|make4ht|. You may need
to include the \verb|-| character in the place of filename for the correct
execution.

For example, if you want to use the \verb|fn-in| option, use the following command:

\begin{verbatim}
$ texblend  -H -O '- "fn-in"' sample.tex
\end{verbatim}

If you want to use for example a configuration file, use

\begin{verbatim}
$ texblend -l -H -O '-c config.cfg - "fn-in"' sample.tex
\end{verbatim}
