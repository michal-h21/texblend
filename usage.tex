% !TEX root = texblend-doc.tex
% !TEX TS-program = lualatex
\section{Usage}

The basic usage of \TeX Blend is as follows:

\begin{verbatim}
$ texblend chapter.tex
\end{verbatim}

This command combines the text from the document \texttt{chapter.tex} with the preamble
from the main document specified in the root directive and directly compiles
the resulting document. The output will be saved as \texttt{chapter.pdf}.

\subsection{Command Line Options}

TeXBlend supports a variety of command-line options:

\begin{verbatim}
 texblend [options] filename

Available options:
-e,--expand                   Expand variables in the template
-h,--help                     Print the help message
-H,--HTML                     Compile to HTML using TeX4ht
-l,--lualatex                 Select lualatex as compiler 
-o,--output (default "")      Set the output file name
-O,--texoptions (default "")  Extra options that should be passed 
                              to the LaTeX compiler
-p,--pdflatex                 Select pdflatex as compiler 
-P,--print                    Don't compile the document, 
                              print the expanded template instead
-s,--shell-escape             Enable execution of external 
                              commands in LaTeX
-t,--template (default "")    Use explicit template instead of 
                              the one provided in the input file
-v,--version                  Print version info
-x,--xelatex                  Select xelatex as compiler 
<filename> (string)           Use "-" if you want to pass 
                              document from the standard input
\end{verbatim}

You can join multiple short options, for example the following call uses separate short options:

\begin{verbatim}
$ texblend -e -l -o test sample.tex
\end{verbatim}

It can be shortened to this form:

\begin{verbatim}
$ texblend -elo test sample.tex
\end{verbatim}



\subsection{Compilation Options}

You can override the \verb'program' directive, which specifies the \LaTeX\ engine
used, with one of the options \verb|--lualatex|, \verb|--xelatex|, or \verb|--pdflatex|. 
The default program used when the document lacks a \verb'program' directive and none of these
options is used is \verb|pdflatex|.

If the document uses external commands, such as with the Minted or Imakeidx
packages, you can utilize the \verb|--shell-escape| option.

You can specify additional options for the LaTeX compiler using the \verb|--texoptions| option.

Instead of compiling, you can print the assembled document to standard output
using the option \verb|--print|.

To change name of the generated PDF file, use the \verb|--output| option.


\subsection{Usage of Templates}


Instead of the main document specified in the \verb'root' directive, you can use
another document as a template, chosen using the \verb|--template| option. You
can use the entire file path or simply specify the file name, making use of the
\verb'kpsewhich' command (see section~\ref{sec:placement}) to locate the file.

If you are using variables in the template (see section~\ref{sec:variables}), use the
\verb'--expand' option to expand them with the values specified in the document's
directives.




\subsection{HTML Support}

\TeX Blend supports direct export to HTML using \TeX4ht. A note regarding the
\verb|--texoptions| command line option: you can utilize this option to pass all
command-line arguments to \verb|make4ht|. Ensure to include the \verb|-| character in place
of the filename for proper execution.

For example, if you want to use the \verb|fn-in| option for endnotes, use the following command:

\begin{verbatim}
$ texblend  --html --texoptions '- "fn-in"' sample.tex
\end{verbatim}

If you want to use a configuration file in addition, you will need this command:

\begin{verbatim}
$ texblend --html --texoptions '-c config.cfg - "fn-in"' sample.tex
\end{verbatim}
