% !TEX root = texblend-doc.tex
% !TEX TS-program = lualatex

\section{Introduction}
This tool compiles individual files that are included as parts of larger documents. 
It utilizes the preamble of the main document but disregards all other included files.

The main purpose is to allow fast compilation of particular chapters or sections, 
eliminating the need to recompile the entire document.
This facilitates an efficient way to check for formatting or syntax errors in
the particular part of the document being worked on.

\section{Structure of \TeX\ Files}

The basic usage is as follows: consider a main document that includes files for
individual chapters, named for example \texttt{main.tex}:


\begin{verbatim}
\documentclass{article}
\author{John Doe}
\title{Sample Document}
\usepackage{upquote}
\usepackage{microtype}
\usepackage{hyperref}
\begin{document}
\maketitle
\tableofcontents
% !TEX root = texblend-doc.tex
% !TEX TS-program = lualatex

hello world

\section{Usage}



\subsection{HTML Support}

\TeX4ht note about the \verb|--texoptions| command line option: you can use
this option to pass all command line arguments to \verb|make4ht|. You may need
to include the \verb|-| character in the place of filename for the correct
execution.

For example, if you want to use the \verb|fn-in| option, use the following command:

\begin{verbatim}
$ texblend  -H -O '- "fn-in"' sample.tex
\end{verbatim}

If you want to use for example a configuration file, use

\begin{verbatim}
$ texblend -l -H -O '-c config.cfg - "fn-in"' sample.tex
\end{verbatim}

\end{document}
\end{verbatim}

We can compile this document to obtain a PDF file containing the text of all
chapters. However, by using \TeX Blend, we can generate separate PDFs for each
chapter. The advantage is that all commands defined in the packages used in the
main document will be available.

In the included files, we can use special directives that indicate the main
file and the engine to be used for compilation. Similar directives are employed
by various \TeX editors, such as \TeX Shop or \TeX Works.

For instance, a file like \texttt{intro.tex}, which is included in the main document,
might look like this:

\begin{verbatim}
% !TEX root = main.tex
% !TEX TS-program = lualatex

\section{Introduction}
This tool compiles individual files that are included 
as parts of larger documents. 
\end{verbatim}

Notice the comments at the beginning of the file. These are known as
directives, indicating the main document and the engine to be used for
compilation. Many \TeX\ editors support these directives, and we utilize them in
\TeX Blend to effectively control the compilation process too.

These directives follow a specific format, requiring placement at the beginning
of the file and commencing with a percentage sign. Subsequently, there is a
space followed by either the marker \texttt{!TEX} or \texttt{!TeX}, 
both of which are recognized formats. 
The next element is the property name with an optional \texttt{TS-} prefix 
(in line with the TeX Shop editor), followed by an equals sign and, finally, the
value.

Two crucial properties are \texttt{root}, indicating the name of the main file, and
\texttt{program}, through which we can select the program for compiling the document.
However, it is possible to specify additional properties, which can then be
utilized in template documents.

\subsection{Placement of the Main Document and Templates}

You can specify the full path to the main document in the \verb|root| property.
\TeX Blend can also be used to compile documents using a universal template. 
The path to this template can be specified in full or can leverage the directory
structure of TeX by placing it in a directory recognized by the KPSE library.
This directory is typically designated for user packages, commonly found at
\verb|~/texmf/tex/latex|, although it may vary on different systems. 
You can identify its location on your system using the command:

\begin{verbatim}
$ kpsewhich --var-value TEXMFHOME
\end{verbatim}

This command typically outputs a path similar to this: \texttt{/home/username/texmf}.
If this directory does not exist, create it. Further, within this directory,
create the subdirectories \texttt{tex/latex/templates}:

\begin{verbatim}
$ mkdir -p ~/texmf/tex/latex/templates
\end{verbatim}

You can then place your templates in this directory. 
To test the availability of the template, run the following command:

\begin{verbatim}
$ kpsewhich templatename.tex
\end{verbatim}

It should print path of the template document.

\subsection{Custom Variables in Templates}

As templates are intended to be universal across various projects, it is
possible to incorporate variables into them. These variables can be populated
from properties in the directives, such as the document title.

To include variables for expansion in the template, you can use the notation
\verb|{{variableName}}|. For example, to set the document title, you can use
\texttt{\textbackslash title\{\{\{title\}\}\}}. The double curly braces around the variable will be replaced
with its value, while the third pair of curly braces will remain in the
document.

Say that you have a following template saved in \url{~/texmf/tex/latex/templates/mytemplate.tex}:

\begin{verbatim}
\documentclass{article}
\title{{{title}}}
\author{Michal}
\usepackage{microtype}
\begin{document}
\end{document}
\end{verbatim}


A document that utilizes this template may look like the following:

\begin{verbatim}
% !TEX root = mytemplate.tex
% !TEX title = hello world
\maketitle
\end{verbatim}



\subsection{Template Documents}



